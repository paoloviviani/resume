%%%%%%%%%%%%%%%%%
% This is an sample CV template created using altacv.cls
% (v1.1.1, 7 December 2016) written by LianTze Lim (liantze@gmail.com). 
%% Now compiles with both XeLaTeX and LuaLaTeX.
%
%% It may be distributed and/or modified under the
%% conditions of the LaTeX Project Public License, either version 1.3
%% of this license or (at your option) any later version.
%% The latest version of this license is in
%%    http://www.latex-project.org/lppl.txt
%% and version 1.3 or later is part of all distributions of LaTeX
%% version 2003/12/01 or later.
%%%%%%%%%%%%%%%%

%% If you need to pass whatever options to xcolor
\PassOptionsToPackage{dvipsnames}{xcolor}

%% If you are using \orcid or academicons
%% icons, make sure you have the academicons
%% option here, and compile with XeLaTeX
%% or LuaLaTeX.
% \documentclass[10pt,a4paper,academicons]{altacv}
\documentclass[10pt,a4paper]{altacv}

\newcommand{\mycvachievement}[2]{%
\begin{itemize}[align=left,labelsep=5mm,itemsep=4mm,itemindent=-5mm,noitemsep,topsep=0pt]
\item[{\Large\color{accent}#1}] \parbox[c]{\linewidth}{\bfseries\textcolor{emphasis}{#2}}
\end{itemize}
}


%% AltaCV uses the fontawesome and academicon fonts
%% and packages.
%% See texdoc.net/pkg/fontawecome and http://texdoc.net/pkg/academicons for full list of symbols.
%%
%% Compile with LuaLaTeX for best results. If you
%% want to use XeLaTeX, you may need to install
%% Academicons.ttf in your operating system's font
%% folder.


% Change the page layout if you need to
\geometry{left=1cm,right=9cm,marginparwidth=6.8cm,marginparsep=1.2cm,top=1.25cm,bottom=1.25cm}

% Change the font if you want to.

% If using pdflatex:
\usepackage[utf8]{inputenc}
\usepackage[T1]{fontenc}
\usepackage[default]{lato}
\usepackage{hyperref}

% If using xelatex or lualatex:
% \setmainfont{Lato}

% Change the colours if you want to
\definecolor{Mulberry}{HTML}{72243D}
\definecolor{SlateGrey}{HTML}{2E2E2E}
\definecolor{LightGrey}{HTML}{666666}
\definecolor{SteelBlue}{HTML}{4682B4}
\definecolor{DarkSteelBlue}{HTML}{24435c}
\definecolor{RifleGreen}{HTML}{414833}
\definecolor{IndiaGreen}{HTML}{138808}

% Green theme
%\colorlet{heading}{RifleGreen}
%\colorlet{accent}{IndiaGreen}

% Red wine theme
%\colorlet{heading}{Sepia}
%\colorlet{accent}{Mulberry}

% Blue theme
\colorlet{heading}{DarkSteelBlue}
\colorlet{accent}{SteelBlue}
\colorlet{ACCENT}{DarkSteelBlue}

% common colors
\colorlet{emphasis}{SlateGrey}
\colorlet{body}{LightGrey}

% Change the bullets for itemize and rating marker
% for \cvskill if you want to
\renewcommand{\itemmarker}{{\small\textbullet}}
\renewcommand{\ratingmarker}{\faCircle}

% SQUARE PHOTO ***
\makeatletter
\patchcmd{\makecvheader}
{\tikz\path[fill overzoom image={\@photo}]circle[radius=0.5\linewidth];}  % original definition
{\includegraphics[width=\linewidth]{\@photo}}   % replace it with this instead
{}{}
\makeatother
% ***

%% sample.bib contains your publications
\addbibresource{sample.bib}

\begin{document}
\name{Paolo Viviani}
\tagline{PhD, Research Engineer}
\photo{2cm}{foto2}
\personalinfo{%
  % Not all of these are required!
  % You can add your own with \printinfo{symbol}{detail}
  % \printinfo{\faUser}{June 14th, 1989}  
  \email{paolo.vivi@gmail.com}
  \phone{+39 329 1865014}
  %\mailaddress{Via Arquata 7, 10134 Torino}
  \location{Torino, ITALY} 
  \homepage{\url{paoloviviani.github.io}}
  %\twitter{@twitterhandle}
  %\linkedin{linkedin.com/in/yourid}
  %\github{github.com/yourid}
  %% You MUST add the academicons option to \documentclass, then compile with LuaLaTeX or XeLaTeX, if you want to use \orcid or other academicons commands.
%   \orcid{orcid.org/0000-0000-0000-0000}
}

%% Make the header extend all the way to the right, if you want. Extend the right margin by 8cm (=6.8cm marginparwidth + 1.2cm marginparsep)
\begin{adjustwidth}{}{-8cm}
\makecvheader
\end{adjustwidth}

%% Provide the file name containing the sidebar contents as an optional parameter to \cvsection.
%% You can always just use \marginpar{...} if you do
%% not need to align the top of the contents to any
%% \cvsection title in the "main" bar.

\cvsection[sample-p1sidebar]{\printinfo{\faInstitution}{Education}}

\cvevent{M.Sc.\ in Theoretical physics}{University of Torino, 104/110}{2015}{Collegio Universitario ``R. Einaudi'', Torino}{
Scholarship winner: "Piano Lauree Scientifiche 2008", granted by Società Italiana di Fisica.}

\divider

\cvevent{Ph.D. in Computer Science}{University of Torino, Computer Science Dept.}{2015 -- 2019}{Torino, IT}{Thesis:
\textit{Deep Learning at Scale with Nearest Neighbours Communications}.\\Supervisor: Marco Aldinucci. Funded by Noesis Solutions.}
\smallskip
\begin{itemize}
\item Co-supervisor of one M.Sc. Thesis
\end{itemize}

\cvsection{\printinfo{\faIndustry}{Experience}}

\cvevent{Noesis Solutions}{Reasearch Engineer}{April 2015 -- Permanent}{Novara, IT}
\begin{itemize}
\item Development of machine learning modelling methodologies for engineering.
\item Re-definition of the internal development and deployment stack (C++, Python) for scientific computations, with focus on performance and maintainability.
\item Cloud and container technologies exploitation.
\item Supervision of one internship.
\item Re-designed internal source code management workflow.
\item Technical contact for the following research projects. Including presentation to European Commission reviewers.
\end{itemize}

\cvsubsection{Funded Research Projects}

\cveventlinkshort{MACH -- ITEA2 project}{April 2015 -- December 2016}{\href{https://itea3.org/project/mach.html}{itea3.org/project/mach.html}}
%MAssive Calculations on Hybrid systems. The goal of the project is to develop a DSeL and a computation framework that allows to access hybrid hardware acceleration without specific expertise.


\smallskip

\cveventlinkshort{Fortissimo 2 -- H2020-FoF project}{February 2016 -- October 2018}{\href{https://www.fortissimo-project.eu/}{www.fortissimo-project.eu}}
%FF2 is a collaborative project that will enable European SMEs to be more competitive globally through the use of simulation services running on a High Performance Computing cloud infrastructure.

\smallskip

\cveventlinkshort{CloudFlow -- FP7-I4MS project}{March 2016 -- May 2017}{\href{http://www.eu-cloudflow.eu/}{www.eu-cloudflow.eu}}
%Cloudflow will enable the remote use of computational services distributed on the cloud, seamlessly integrating these within established engineering design workflows and standards.

\smallskip

\cveventlinkshort{Blockchain for online Service Security -- IMEC-ICON Flemish project}{March 2018 -- November 2019}{}

\smallskip

\cveventlinkshort{VaProFam -- FlandersMake-ICON Flemish project}{February 2019 -- ongoing}{}

\smallskip

\cveventlinkshort{DC-CDS -- FlandersMake-ICON Flemish project}{April 2019 -- ongoing}{}

\clearpage

\cvsection[sample-p2sidebar]{\printinfo{\faBook}{Academic}}

\renewcommand*{\bibfont}{\small}
\cvsubsection{Publications}
\nocite{*}
\smallskip
\printbibliography[heading=none]

\cvsubsection{Other}
\begin{itemize}
\item Program Committee member, \textit{Euromicro International Conference on Paralle, Distributed, and Network-based Processing (PDP)} for 2018 (also session chair), 2019 and 2020.
\item Program Committee member, \textit{Parallel Numerical Methods and Libraries for Heterogeneous Multi/Manycores (Special Session of PDP2018 and PDP2019)}
\item Program Committee member, \textit{Artifact Evaluation, Euro-Par 2018 Conference}
\item Program Committee member, \textit{16th IEEE International Conference on Scalable Computing and Communications (ScalCom 2016)}
\end{itemize}

\end{document}