\newcommand{\mycvachievement}[2]{%
\begin{itemize}[align=left,labelsep=5mm,itemsep=4mm,itemindent=-5mm,noitemsep,topsep=0pt]
\item[{\Large\color{accent}#1}] \parbox[c]{\linewidth}{\bfseries\textcolor{emphasis}{#2}}
\end{itemize}
}

\newcommand{\cvprojectsmall}[2]{%
  \textbf{\color{accent}#1}\par
  \smallskip
  {\small\makebox[0.5\linewidth][l]{\faCalendar \hspace{0.5em}#2}}
  \medskip
}

\newcommand{\cveventlinkshort}[4]{%
  \textbf{\color{accent}#1}\par
  \smallskip
  {\small\makebox[0.5\linewidth][l]{\faCalendar \hspace{0.5em}#2}%
  \ifstrequal{#3}{}{}{\makebox[0.5\linewidth][l]{\faChain\hspace{0.5em}#3}}\par}
  \medskip
}

\newcommand{\cvaccentpar}[1]{%
  \smallskip%
  \textbf{\color{accent}#1}\par
  \medskip
}

\newcommand{\cvproject}[3]{%
  \smallskip%
  \parbox[t]{0.25\linewidth}{\textbf{\color{accent}#1}}%
  {\small\makebox[0.4\linewidth][l]{\faUniversity \hspace{0.5em}#2}}%
  {\ifstrequal{#3}{}{}{\small\makebox[0.3\linewidth][l]{\faCalendar \hspace{0.5em}#3}}\par}%
  \medskip
}

\newcommand{\cveventlogo}[5]{%
\begin{minipage}{0.6\textwidth}
  {\large\color{emphasis}#2\par}
  \smallskip
  \textbf{\color{accent}#3}\par
  \medskip
\end{minipage} \hfill
\begin{minipage}{0.2\textwidth}
  \hfill
  \includegraphics[height=1cm,width=2cm,keepaspectratio]{#1}
  \medskip
\end{minipage}  
{\small\makebox[0.4\linewidth][l]{\faCalendar \hspace{0.5em}#4}%
\ifstrequal{#5}{}{}{\makebox[0.5\linewidth][l]{\faMapMarker\hspace{0.5em}#4}}\par}
\medskip
}

%% AltaCV uses the fontawesome and academicon fonts
%% and packages.
%% See texdoc.net/pkg/fontawecome and http://texdoc.net/pkg/academicons for full list of symbols.
%%
%% Compile with LuaLaTeX for best results. If you
%% want to use XeLaTeX, you may need to install
%% Academicons.ttf in your operating system's font
%% folder.

% Change the page layout if you need to
\geometry{left=1cm,right=9cm,marginparwidth=6.8cm,marginparsep=1.2cm,top=1.25cm,bottom=1.25cm}

% Change the font if you want to.

% If using pdflatex:
\usepackage[utf8]{inputenc}
\usepackage[T1]{fontenc}
\usepackage[default]{lato}
\usepackage[hidelinks]{hyperref}

%colored boxes
\usepackage[absolute,noshowtext]{textpos}

\usepackage{multicol}
\setlength{\multicolsep}{1.0pt plus 2.0pt minus 1.5pt}% 50% of original values

% If using xelatex or lualatex:
% \setmainfont{Lato}

% Change the colours if you want to
\definecolor{Mulberry}{HTML}{72243D}
\definecolor{SlateGrey}{HTML}{2D2D2D}
\definecolor{LightGrey}{HTML}{666666}
\definecolor{SteelBlue}{HTML}{4682B4}
\definecolor{DarkSteelBlue}{HTML}{24435c}
\definecolor{RifleGreen}{HTML}{414833}
\definecolor{IndiaGreen}{HTML}{138808}
\definecolor{offwhite}{HTML}{FFFAFA}
\definecolor{VeryLightGrey}{HTML}{eaeaea}
%grey-blue-orange
\definecolor{gbo-blue}{HTML}{465359}
\definecolor{gbo-Grey}{HTML}{969fa7}
\definecolor{gbo-orange}{HTML}{eb8436}

% Green theme
%\colorlet{heading}{RifleGreen}
%\colorlet{accent}{IndiaGreen}

% Red wine theme
%\colorlet{heading}{Sepia}
%\colorlet{accent}{Mulberry}

% Blue theme
\colorlet{heading}{DarkSteelBlue}
\colorlet{accent}{SteelBlue}
\colorlet{ACCENT}{SteelBlue}

% common colors
\colorlet{emphasis}{SlateGrey}
\colorlet{body}{LightGrey}
%\colorlet{negativeemph}{offwhite}
\colorlet{negativeemph}{SlateGrey}

% Change the bullets for itemize and rating marker
% for \cvskill if you want to
\renewcommand{\itemmarker}{{\small\textbullet}}
\renewcommand{\ratingmarker}{\faCircle}

% SQUARE PHOTO ***
%\makeatletter
%\patchcmd{\makecvheader}
%{\tikz\path[fill overzoom image={\@photo}]circle[radius=0.5\linewidth];}  % original definition
%{\includegraphics[width=\linewidth]{\@photo}}   % replace it with this instead
%{}{}
%\makeatother
% ***
\usepackage{ragged2e}

